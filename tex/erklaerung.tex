% erklaerung.tex
\cleardoublepage
\newgeometry{left=2.7cm,top=1.4cm,right=2.7cm,bottom=2cm}
\pagestyle{plain}
\begin{center}
	\Huge
	\sffamily
	\textbf{Eidesstattliche Versicherung}
\end{center}
\vspace{1cm}
\normalsize
\begin{tabular}{c}
\textbf{Stahl, Christoph}\\
Name, Vorname
\end{tabular}\hfill 
\begin{tabular}{c}
	\textbf{116531}\\
Matr.-Nr.
\end{tabular}\\
\vspace*{1cm}\\
Ich versichere hiermit an Eides statt, dass ich die vorliegende Masterarbeit mit dem Titel 

\begin{center}
\begin{minipage}{.5\textwidth}
\begin{center}
	\textbf{Prinzipale Inhabitation im einfach getypten Lambda-Kalkül}
\end{center}
\end{minipage}
\end{center}

\noindent selbstständig und ohne unzulässige fremde Hilfe erbracht habe. Ich habe keine anderen als die angegebenen Quellen und Hilfsmittel benutzt sowie wörtliche und sinngemäße Zitate kenntlich gemacht. Die Arbeit hat in gleicher oder ähnlicher Form noch keiner Prüfungsbehörde vorgelegen. 


\begin{tabular}{c}
	\vspace{4em}\hspace*{4cm}\\
	\hline
	Ort, Datum
\end{tabular}
\hfill
\begin{tabular}{c}
	\vspace{4em}\hspace*{4cm}\\
\hline
	Unterschrift
\end{tabular}\\

\noindent\textbf{Belehrung: }

\noindent Wer vorsätzlich gegen eine die Täuschung über Prüfungsleistungen betreffende Regelung einer Hochschulprüfungsordnung verstößt, handelt ordnungswidrig. Die Ordnungswidrigkeit kann mit einer Geldbuße von bis zu 50.000,00 \euro{} geahndet werden. Zuständige Verwaltungsbehörde für die Verfolgung und Ahndung von Ordnungswidrigkeiten ist der Kanzler/die Kanzlerin der Technischen Universität Dortmund. Im Falle eines mehrfachen oder sonstigen schwerwiegenden Täuschungsversuches kann der Prüfling zudem exmatrikuliert werden. (§ 63 Abs. 5 Hochschulgesetz - HG - )  

\noindent Die Abgabe einer falschen Versicherung an Eides statt wird mit Freiheitsstrafe bis zu 3 Jahren oder mit Geldstrafe bestraft.  

\noindent Die Technische Universität Dortmund wird gfls. elektronische Vergleichswerkzeuge (wie z.B. die Software \emph{turnitin}) zur Überprüfung von Ordnungswidrigkeiten in Prüfungsverfahren nutzen. 

\noindent Die oben stehende Belehrung habe ich zur Kenntnis genommen:\\
\begin{tabular}{c}
	\vspace{4em}\hspace*{4cm}\\
	\hline
	Ort, Datum
\end{tabular}
\hfill
\begin{tabular}{c}
	\vspace{4em}\hspace*{4cm}\\
\hline
	Unterschrift
\end{tabular}\\
% EOF
